
\begin{domanda}
Sia $p$ un punto in cui l'hessiano della funzione $f$  a due variabili è definito positivo.

Quale tra le seguenti è corretta:
\begin{itemize}
\item se $f$ ha differenziale non nullo in $p$, allora $p$ può essere un minimo locale: ha senso, il differenziale/gradiente non è nullo e quindi posso fare la derivata seconda che mi permette di definire il segno
\end{itemize}

\end{domanda}

\begin{domanda}
Una funzione $f$ in due variabili $x,y$ ammette derivate parziali in un dominio $D$ del piano se:

\item se è integrabile su $D$: ???
\item se è continua su $D$: la continuità non implica nulla in $R^2$
\item se è quoziente di due polinomi, uno strettamente positivo e uno strettamente negativo su $D$: si ottiene una funzione continua e come abbiamo detto la continuità non implica nulla.
\item ammette almeno due derivate direzionali in direzioni distinte in ogni punto di $D$: ??? 
\end{domanda}

\begin{domanda}
Si consideri la funzione $f(x,y)=2x^3y^2-x^2y$. Supponiamo di essere in posizione $P=(1,1,1)$ sul grafico di $f$. Quale angolo (rispetto al semiasse positivo $x$) il mio vettore spostamento deve seguire perchè io non scenda, immediatamente dopo la partenza?

E' il problema fatto in classe dell'omino che scappa dal fuoco.

Per prima cosa individuiamo il gradiente nel punto.

$\nabla f = (6x^2y^2-2xy,4yx^3-x^2)$
$\nabla f(1,1) = (4,3)$

    
\end{domanda}