# Matrici

Una matrice con *p* righe e *q* colonne è una tabella di numeri reali così disposti:

$$
A =
\begin{bmatrix}
    a_{11} & a_{12} & a_{13} & \dots  & a_{1q} \\
    a_{21} & a_{22} & a_{23} & \dots  & a_{2q} \\
    \vdots & \vdots & \vdots & \ddots & \vdots \\
    a_{p1} & a_{p2} & a_{p3} & \dots  & a_{pq}
\end{bmatrix}
$$

$p$ e $q$ sono detti dimensioni della matrice.

L'elemento A_{i,j} della matrice è l'elemento che si trova alla $i$-esima riga e alla $j$-esima colonna.

## Matrici quadrate

Una matrice che ha dimensione $(n,n)$ è detta matrice quadrata. Questa matrice avrà un numero uguale di righe e di colonne.
$$
A_{\text{quadrata}} =
\begin{bmatrix}
    a_{11} & a_{12} \\
    a_{21} & a_{22}
\end{bmatrix}
$$

### Diagonale principale

Per ogni matrice quadrata $A_{n\times n}$ è possibile individuare gli elementi della diagonale principale, cioè tutti gli $a_{i,i}$ con $i$ che varia da 1 a $n$. 

### Matrice triangolare superiore
Una matrice quadrata $A_{n\times n}$ si dice triangolare superiore se tutti gli elementi che si trovano sotto la diagonale principale sono nulli.

$$
A_{\text{diag. sup.}} =
\begin{bmatrix}
    1 & 2 & 3 \\
    0 & 4 & 5 \\
    0 & 0 & 6 \\
\end{bmatrix}
$$

### Matrice diagonale
Una matrice quadrata $A_{n \times n}$ è detta diagonale se tutti gli elementi non appartenenti alla diagonale principale sono nulli.

### Matrice simmetrica 
Una matrice quadrata si dice simmetrica se i suoi elementi in posizioni simmetriche rispetto alla diagonale principale sono uguali.


### Matrice identità
Chiamiamo matrice unità o matrice identica di ordine $n$ la matrice quadrata $I_{n \times n}$ avente tutti gli elementi della diagonale principale uguali a 1 e tutti gli altri elementi uguali a 0. 

La matrice identica è una matrice diagonale.


## Operazioni con matrici

### Matrice trasposta
Presa una matrice $A$ chiamiamo $A^t$ la trasposta di $A$ la matrice avente come elemento di posto $(i, j)$ l’elemento $(j, j)$ della matrice A.

$$A =
\begin{bmatrix}
    1 & 2 & 3\\
    4 & 5 & 6
\end{bmatrix}

\\

A^t = \begin{bmatrix}
    1 & 4 \\
    2 & 5 \\
    3 & 6
\end{bmatrix}
$$

### Somma tra matrici

Consideriamo due matrici $A$ e $B$. Le due matrici sono sommabili se e solo se sono dello stesso tipo, cioè se e solo se hanno lo stesso numero di righe e di colonne. 

$C=A+B$ è una matrice avente lo stesso numero di righe e di colonne delle due matrici di partenza, in cui ogni termine $C_{i,j}=A_{i,j}+B_{i,j}$.



 
$$
A = \begin{bmatrix} 
    2 & 5 & -3 \\
    1 & -2 & 4  
\end{bmatrix}

B = \begin{bmatrix} 
    7 & -5 & 2 \\
    -9 & 4 & -1  
    \end{bmatrix}
$$

$$
C = A+B = \begin{bmatrix} 
    2+7 & 5+(-5) & (-3)+2 \\
    (-9)+1 & (-2)+4 & 4+(-1)  
    \end{bmatrix}
    =
    \begin{bmatrix} 
    9 & 0 & -1 \\
    -8 & 2 & 3  
    \end{bmatrix}
$$

### Moltiplicazione per uno scalare
La moltiplicazione di una matrice $A=(a_{i,j})$ per uno scalare $r$ è ottenuta moltiplicando ogni elemento di $A$ per lo scalare:
$$ rA = (ra_{i,j}) $$

### Prodotto tra matrici
Siano $A, B$ due matrici. E' possibile calcolare il prodotto $C=AB$ solo se righe($A$) = colonne($B$) e colonne($A$) = righe($B$)

$C_{ij} := A_{i1}B_{1j} + A_{i2}B_{2j} + \ldots + A_{iq}B_{qj}$


### Inversa di una matrice

### Rango di una matrice

Il rango di una matrice è il massimo ordine di sottomatrice quadrata con determinante diverso da 0 che posso estrarre dalla magrice stessa.

Il rango di una matrice ridotta a scala è il numeri di righe diverse da 0.


## Il determinante delle matrici
* Sarrus (solo 3x3)
* Calcolo per matrici ridotte
    * Su riga o colonna

### Proprietà
Una matrice quadrata con due righe o due colonne uguali ha determinante nullo.

Siano A e B due matrici quadrate di ordine n che si ottengono una dall’altra scambiando fra loro due righe. Allora det A = − det B. Un’analoga proprietà vale per lo scambio di colonne.

Se una matrice quadrata A ha una riga che è multipla di un’altra, allora det A = 0. Un’analoga proprietà vale per le colonne.

Sia A una matrice quadrata di ordine n e k un numero reale. Si ha allora: $\det (kA) = k^n \det A$.

**Teorema di Binet** Date due matrici quadrate dello stesso ordine A e B si ha: $\det(AB) = \det A \det B$.


# Sistemi lineari

## Teoria

### Teorema di Cramer

Se $\det(A) \ne 0$, allora c'è una ed una sola soluzione, altrimenti nessuna o infinite.

## Metodi risolutivi

### Cramer

Solo se la matrice delle incognite è quadrata

A_x = \frac{det A(x)}{det A}
A_y = \frac{det A(y)}{det A}
A_z = \frac{det A(z)}{det A}

### Gauss

Si riduce a scala la matrice e si sostituisce

Operazioni permesse: 
* Scambio di righe e colonne (se scambio le colonne devo fare attenzione)
* Moltiplico per uno scalare una riga/colonna
* Scrivo una riga come $k * \text{riga} + q * \text{altraRiga}$

### Gauss Jordan

Solo se la matrice delle incognite è quadrata, mi serve calcolare l'inversa.

$Ax=b \rightarrow A\cdot x \cdot A^{-1} = b \cdot A^{-1} \rightarrow I\cdot x = b \cdot A^{-1}$

# Vettori

## Dipendenza lineare

$n$ vettori sono linearmente dipendenti se e solo se il determinante della matrice associata è 0.

Il rango rappresenta il numero massimo di righe/colonne linearmente indipendenti. Posso eliminare dal sistema le righe che non sono nel rango.

Se numero incognite > numero equazioni ci sono infinito^(n-Rango) soluzioni.

Se $rk(A|\underline{b})=rk(A)=n$, allora abbiamo una ed una sola soluzione;

Se $rk(A|\underline{b})=rk(A)<n$, allora abbiamo $\infty^{n-rk(A)}$ soluzioni.

