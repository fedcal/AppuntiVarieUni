
\chapter{La pubblicità}
	Il modello di business classico è dare servizio gratis attraverso il web, ricevere utenti e sostenersi con la pubblicità. Piccolo intoppo: \textbf{gli utenti odiano la pubblicità}. Solo il 0,4\% clicca sulle pubblicità. Cosa fare per migliorare questa situazione e sopratutto far sì che l'utente clicchi su essa?
	\begin{itemize}
		\item Buon posizionamento banner.
		\item Messaggio efficace, bello e attraente.
	\end{itemize}

	\section{Posizionamento}
		Per quanto riguarda il posizionamento del o dei banner in coordinate assolute, per ordine di utilità, abbiamo:
		\begin{enumerate}
			\item Colonna di sinistra.
			\item Top della pagina.
			\item Colonna di destra.
			\item Bottom (non serve praticamente a nulla).
		\end{enumerate}
		Altri accorgimenti da adottare possono essere:
		\begin{itemize}
			\item Posizionare la pubblicità vicino al contenuto interessante così, maggiori visualizzazioni per essa.
			\item Rispettare le taglie minime e tener conto che la taglia non influenza molto. I banner grandi non attraggono molto di più di quelli piccoli.
		\end{itemize}
	
	\section{Messaggio efficace}
		Prima di analizzare come fare un messaggio pubblicitario efficace mostriamo le \textbf{11 cose} da \textbf{non fare} con percentuale di insoddisfazione generata agli utenti.
		
		\subsection{Top 11 disgrazie}
			\begin{etaremune}
				\item Suona automaticamente (79\%).
				\item È in movimento (79\%).
				\item Lampeggia (87\%).
				\item Occupa la maggior parte della pagina (90\%).
				\item Si sposta sullo schermo (92\%).
				\item Non dice di cosa si tratta (92\%).
				\item Copre il contenuto da leggere (93\%).
				\item Non dispone di un modo chiaro per chiudersi (93\%).
				\item È qualcosa che cerca di farti cliccare sopra (94\%).
				\item Si carica lentamente (94\%).
				\item È un pop-up (95\%).
			\end{etaremune}
		
		\subsection{Il mondo pubblicitario nel web}
			Per risaltare il messaggio nel mondo pubblicitario esistono accorgimenti molto efficaci come, ad esempio:
			\begin{itemize}
				\item Utilizzare persone belle per attirare l'attenzione.
				\item Usare colori vivaci ed effetti speciali.
			\end{itemize}
			Questi accorgimenti trasportanti nel web rimarranno efficaci?
		
			\subsubsection{Effetto zapping}
				Ricordiamo che le immagini nel web godono di meno visibilità e attenzione da parte dell'utente rispetto al testo, questo deriva da un processo automatico e subconscio (effetto \emph{zapping} nel web) che salta le immagini e soprattutto i banner. Portare infatti un contenuto in forma simile a immagini e banner può far sì che questo non sia nemmeno visualizzato dalla maggior parte degli utenti. 
				
				Gli utenti sono abituati a ritenere superflue le informazioni contenute nei banner per cui durante la navigazione il cervello utilizza dei filtri (che utilizziamo quotidianamente anche in altre situazioni) sull'informazione entrante dall'apparato visivo.		
			
		\subsection{Immagini in serie A}
			L'unico modo, paradossale, per rendere attrattive le nostre immagini e quindi banner è \textbf{andare controcorrente}. L'algoritmo abituale per filtrare è interrotto da qualcosa di inusuale e l'utente è costretto a porre maggiore attenzione. Quindi:
			\begin{itemize}
				\item Niente colori vistosi e attraenti.
				\item Confondere le idee all'utente.
			\end{itemize}
			Per far ciò si può ricorrere ad alcune tecniche:
			\begin{description}
				\item[Blending:] eliminare le zone della pubblicità confondendo pubblicità con contenuto. Il miglior \emph{blending} è il testo.
				\item[Giochetti web:] l'uso di giochi attira l'attenzione e azzera i timer dell'utente finché è impegnato a giocare. Una buona tecnica quindi è quella di inserire pubblicità insieme a giochi.
			\end{description}
			
			Attenzione però al \textbf{\emph{distraction effect}}: la pubblicità deve seguire il contesto in cui è contenuto altrimenti i timer dell'utente diminuiscono del 40\% e la voglia di rivisitare il sito scende addirittura dell'80\%.
		
		\subsection{Il problema del contenuto}
			Per risolvere il problema del contenuto e del \emph{distraction effect} si è ricorso al \emph{behavior advertising}, pubblicità che seguono il comportamento dell'utente nel web e che quindi è più probabile siano interessanti. Per far questo è necessario raccogliere dati su ogni singolo utente. Questo può rendere fino a 10000 volte più efficace la pubblicità e in certi casi ribalta l'insoddisfazione con aumento dei timer e voglia di ritornare al sito.