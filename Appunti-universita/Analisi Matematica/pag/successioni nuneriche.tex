\chapter{Successioni numeriche}
\section{Definizione}
Una successione $\{a_n\}$ è una funzione che ad ogni numero naturale n associa un numero reale $a_n$
\begin{multicols}{2}
	\begin{equation*}
		\{a_n\}: \begin{matrix}
			n&\to&a_n\\
			0&\to&a_0\\
			1&\to&a_1\\
			\vdots&&\vdots\\
			n&\to&a_n
		\end{matrix}
	\end{equation*}
	Esempio:
	\begin{equation*}
		\bigg\{\frac{1}{n}\bigg\}: \begin{matrix}
			n&\to&a_n\\
			1&\to&1\\
			2&\to&\frac{1}{2}\\
			\vdots&&\vdots\\
			n&\to&\frac{1}{n}
		 \end	{matrix}
	\end{equation*}
\end{multicols}
Il limite della successione $\{a_n\}$ è il numero reale a (\textit{si dice anche che $a_n$ converga ad a}) e si indica
\begin{eqnarray*}
	\lim_{n\to\infty}a_n=a&(a_n\to a)
\end{eqnarray*}
se, qualunque sia $\varepsilon>0 \exists v_\varepsilon:$ $|a_n-a|<\varepsilon\text{ } \forall n>v_\varepsilon$. Esempio tramite la definizione dimostriamo che
\begin{equation*}
	\lim_{n\to\infty}\frac{1}{n}=0
\end{equation*}
Si ha $|a_n-a|=\bigg|\frac{1}{n}-0\bigg|=\frac{1}{n}<\varepsilon\Rightarrow n>\frac{1}{\varepsilon}$. Quindi basta porre $v=\frac{1}{\varepsilon}$ e si ha che $\forall n>v\text{ }\lim_{n\to \infty}\frac{1}{n}=0$
\begin{itemize}
	\item Se il limite della successione $\{a_n\}$ è un numero finito allora la successione si dirà convergente (o regolare);
	\item Se il limite di $\{a_n\}$ è infinito, allora si dirà divergente (regolare);
	\item Se invece tale limite non esiste, allora $\{a_n\}$ si dice indeterminata (o irregolare);
\end{itemize}
La definizione di limite per la successione $\{a_n\}$ e i teoremi sui limiti sono analoghi a quelli visti per le funzioni.
\paragraph{Definizione}
\begin{eqnarray*}
	\lim_{n\to+\infty}a_n=+\infty&\Leftrightarrow&\forall K>0 \exists N=N(K):\forall n\geq N:a_n>K\\
	\lim_{n\to+\infty}a_n=-\infty&\Leftrightarrow&\forall H>0 \exists M=M(K):\forall n\geq M:a_n<-H
\end{eqnarray*}
\paragraph{Teorema dell'unicità del limite}
\begin{equation*}
	\text{Se } \lim_{n\to \infty} a_n=a\Rightarrow a \text{ a è unico.}
\end{equation*}
\section{Teorema della permanenza del segno}
Se una successione $\{a_n\}$ converge ad un limite strettamente positivo $a>0$ (che può essere anche $+\infty$), ossia se $\lim_{n\to+\infty}a_n=a>0$ (o $a<0$)\\
Allora $a_n>0$ definitivamente (o $a<0$), ossia ha definitivamente soltanto termini positivi (o negativi). Per le successioni, <<definitivamente>> significa per n abbastanza grande. - In altre parole, esiste un N tale che $a_n>0$ per ogni $n>N$. Esempio $n-10\sqrt{n}$ è definitivamente positiva per $n>100$
\section{Teorema della permanenza del segno}
Una successione che converge a zero può avere infiniti termini di ambo i segni, ad esempio
\begin{eqnarray*}
	a_n=\frac{(-1)^n}{n}
\end{eqnarray*}
Non è vero in generale che una successione $\{a_n\}$ di termini positivi $a_n>0$ convergente debba avere un limite strettamente positivo $a>0$: ad esempio la successione $a_n=\frac{1}{n}$ è fatta di termini positivi, ma converge a zero. Ogni successione $\{a_n\}$ si dice limitata se $\exists M$:
\begin{equation*}
	|a_n|\neq M.
\end{equation*}
Esempio $*\{a_n\}=(-1)^n$ è limitata: $|a_n|=1$ $(M=1)$, ma non ha limite, intatti
\begin{equation*}
	\lim_{n\to +\infty}a_n=\lim_{n\to +\infty}(-1)^n\text{ non esite}
\end{equation*} 
$*\{a_n\}=\sin x$ è limitata ma non ammette limite
\section{Teorema del confronto (o dei due carabinieri)}
Siano $\{a_n\},\{b_n\},\{c_n\}$ tre successioni tali che
\begin{eqnarray*}
	a_n\leq b_n\leq c_n&\forall n\in N.
\end{eqnarray*}
Se $\lim_{n\to \infty} a_n=\lim_{n\to +\infty} c_n=a$, allora anche la successione $b_n$ è convergente e si ha
\begin{equation*}
	\lim_{n\to +\infty}b_n=a.
\end{equation*}
esempio $b_n=\frac{\cos n}{n}$.\\
Se $a_n \leq b_n$ definitivamente, e se $\lim_{n\to+\infty}a_n=+\infty$ allora anche $\lim_{n\to +\infty}b_n=+\infty$, analogamente se $b_n\leq c_n$ e $c_n\to -\infty$ allora anche $b\to -\infty$ - S $\{a_n\}$ è una successione limitata e $\lim_{n\to +\infty}b_n=0$, allora la sccessione prodotto $a_n*b_n\to 0$. 
\begin{eqnarray*}
	\text{esempio }&\lim_{n\to +\infty}\frac{\sin n}{n}=\lim_{n\to+\infty}(\sin n)\frac{1}{n}=0
\end{eqnarray*}
in quanto $\sin n$ è limitata: $|\sin n|\leq 1$ e $\lim_{n\to+\infty}\frac{1}{n}=0$\\
Successioni infinitesime (cioè convergenti a zero) o infinite (cioè divergenti) possono essere confrontabili come si è fatto per le funzioni. Le definizioni sono analoghe.\\
Si ha, anche per le successioni
\begin{equation*}
	\ln n<<n^b<<a^n<<n!<<n^n, b>0, a>1.
\end{equation*}
Anche per le successioni valgono le operazioni con i limiti e le convenzioni con l’$\infty$, visti per le funzioni.
Anche i limiti notevoli visti per le funzioni, si adattano alle successioni - Esempio
\begin{eqnarray*}
	\lim_{n\to+\infty}\frac{\sin\frac{1}{2}}{\frac{1}{n}}=\lim_{n\to+\infty} n\sin\frac{1}{n}=1\\
	\lim_{n\to+\infty}\frac{\cos\frac{1}{2}}{\frac{1}{n^2}}=\lim_{n\to+\infty}\bigg(1-\cos\frac{1}{n}\bigg)=\frac{1}{2}\\
	\lim_{n\to+\infty}\frac{\ln(1+\frac{1}{2})}{\frac{1}{n}}=\ln\bigg(1+\frac{1}{2}\bigg)=1
\end{eqnarray*}
\section{Riassunto}
La successione $\{a_n\}$ si definisce
\begin{eqnarray*}
	\text{monotona \underline{crescente} se}&a_n\leq a_{n+1},&\forall n \in N\\
	\text{monotona \underline{strettamente crescente} se}&a_n<a_{n+1},&\forall n \in N\\
	\text{monotona \underline{decrescente} se}&a_n\geq a_{n+1},&\forall n \in N\\
	\text{monotona \underline{strettamente decrescente} se}&a_n> a_{n+1},&\forall n \in N
\end{eqnarray*}
Ogni successione monotona ammette limite. In particolare ogni successione monotona limitata è convergente (\textit{cioè ammette limite finito: per es $l=\sup a_n$ se $a_n$ è limitata e crescente})
\paragraph{Esercizio. Calcolo del limite:}
\begin{enumerate}
	\item $\lim_{n\to+\infty}\sqrt{n+1}-\sqrt{n}$
	\begin{equation*}
		\lim_{n\to+\infty}\sqrt{n+1}-\sqrt{n}=\lim_{n\to+\infty}\frac{\sqrt{n+1}-\sqrt{n}}{\sqrt{n+1}-\sqrt{n}}(\sqrt{n+1}+\sqrt{n})=\lim_{n\to+\infty}\frac{1}{\sqrt{n+1}+\sqrt{n}}=0
	\end{equation*}
	\item $\lim_{n\to+\infty}\sqrt[n]{n}=\lim_{n\to+\infty}e^{\frac{\ln n}{n}}=e^0=1$
	
\end{enumerate}


