\chapter{programma}
\section{Base}
\begin{itemize}
	\item \textit{Elettrostatica nel vuoto} - carica elettrica, legge di Coulomb, campo elettrico, teorema di Gauss e $1^a$ equazione di Maxwell, potenziale elettrico, dipolo elettrico, conduttori, capacità elettrica, sistemi di condensatori, collegamento in serie e in parallelo, energia del campo elettrostatico.
	\item \textit{Corrente elettrica stazionaria} - resistenza elettrica e legge di Ohm, effetto Joule, forza elettromotrice e generatori elettrici, circuiti in corrente continua.
	\item \textit{Magnetismo nel vuoto} - forza di Lorentz, vettore induzione magnetica, forze magnetica
	 su una corrente, momento magnetico della spira percorsa da corrente, relazione tra momento
	 meccanico e momento magnetico, campi generati da correnti stazionarie, legge di Biot e Savart (campo
	 del filo indefinito, della spira circolare e del solenoide), 2a equazione di Maxwell, teorema di Ampère.
	\item \textit{Campi magnetici variabili nel tempo} - induzione elettromagnetica , legge di 
	Faraday-Newmann, $3^a$ e $4^a$ equazione di Maxwell, autoinduzione, circuito RL, 
	energia magnetica.
	\item \textit{Onde} - equazione d'onda, tipi di onde, velocità di fase, equazioni delle onde
	elettromagnetiche e loro proprietà, onda piana e onde sferiche, energia di un'onda 
	elettromagnetica e vettore di Poynting, spettro della radiazione elettromagnetica. 
\end{itemize}
\section{Argomenti aggiuntivi}
\begin{itemize}
	\item \textit{Elettrostatica nella materia} - la costante dielettrica, interpretazione microscopica, suscettibilità elettrica.
	\item \textit{Magnetismo nella materia} - vettori B, H e M, materiali paramagnetici, ferromagnetici, diamagnetici, legge di Curie, ciclo di isteresi.
\end{itemize}
