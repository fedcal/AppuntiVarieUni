% !TEX encoding = UTF-8
% !TEX TS-program = pdflatex
% !TEX root = ../apprendimento_automatico.tex
% !TEX spellcheck = it-IT
\section{Apprendimento automatico}\label{apprendimento-automatico}

Non sempre è possibile utilizzare degli algoritmi per risolvere un problema.
Per vari motivi:

\begin{itemize}
\item non sempre si può formalizzare un determinato problema
\item ci sono delle situazioni di incertezza
\item risulta troppo complesso trovare una soluzione oppure sono richieste troppe risorse
\end{itemize}

Alcuni esempi sono: riconoscimento facciale, filtro anti-spam.

In questi casi gli algoritmi (sequenza finita di passi che portano ad un risultato determinato in un tempo finito) non funzionano ed è quindi preferibile fornire una soluzione ``\emph{imperfetta}''.

In apprendimento automatico si studiano i metodi per trasformare l'informazione empirica (dati del problema) in conoscenza.

Questo approccio è diventato possibile grazie al fatto che Internet ha reso disponibili molti dati.

\subsection{Le basi}\label{le-basi}

Perché il machine leargning funzioni deve esserci un processo
(stocastico o deterministico) che spiega i dati che osserviamo, in modo
da riuscire a costruire un'approssimazione di tale processo che può
anche risultare imperfetta dal momento che il processo che si vuole
approssimare non è noto.

\emph{Stocastico}: random a probabilità

L'obiettivo finale del machine learning è quello di definire dei criteri
da ottimizzare in modo che sia possibile andare a migliorare dei modelli
definiti su certi parametri.

Questi modelli possono essere:

\begin{itemize}
\item
  \textbf{Preditivi}: per fare previsioni sul futuro (es: filtro
  anti-spam)
\item
  \textbf{Descrittivi}: utilizzare dei dati per ottenere maggiori
  informazioni (data mining)
\end{itemize}

Esempi applicativi:

\begin{itemize}
\item
  Software OCR
\item
  Estrapolazione di dati a partire dal linguaggio naturale
\item
  Riconoscimento facciale
\item
  Giochi con informazione incompleta (Gaist? gioco con fantasmi
  rosso/blu, tedesco)
\end{itemize}

\subsection{Problemi tipici dell'apprendimento automatico}\label{problemi-tipici-dellapprendimento-automatico}

\begin{itemize}
\item
  \textbf{Classificazione binaria}: dato un input dire se appartiene ad
  una determinata classe o meno. Esempio: data una cifre dire se è uno 0
  o meno.
\item
  \textbf{Classificazione multiclasse}: dato un input lo assegno ad una
  determianta categoria. Es: identificare una cifra manoscritta.
\item
  \textbf{Regressione}: dato un insieme di valori, trovare una funzione
  che li approssimi.
\item
  \textbf{Ranking di classi} (non sarà affrontato): data una serie di
  dati, dire quali sono più rilevanti, ovvero, data una serie di
  documenti ordinarli nel modo migliore secondo una determinata
  preferenza, es: motore di ricerca.
\item
  \textbf{Novelty detection}: riconoscimento delle irregolarità a
  partire da una serie di dati. es: frode bancaria su una serie di
  transazioni, controllo degli accessi, ecc.
\item
  \textbf{Clustering}: raggruppamento di dati in modo gerarchico,
  basandosi su alcune caratteristiche che li accomunano o meno.
\item
  \textbf{Associazioni}: quello che fa Amazon con ``altri utenti hanno
  comprato''
\item
  \textbf{Reinforcement Learning}: valutazioni di strategie, quando si
  ha una serie di stati e possibili azioni, si vuole valutare la qualità
  complessiva, es: movimenti di un robot.
\end{itemize}
