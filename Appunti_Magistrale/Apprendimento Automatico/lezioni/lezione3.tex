% !TEX encoding = UTF-8
% !TEX TS-program = pdflatex
% !TEX root = ../apprendimento_automatico.tex
% !TEX spellcheck = it-IT
\section{Lezione 3 - Ripasso di probabilità e algebra + Supervised Learning}\label{lezione-3---ripasso-di-probabilituxe0-e-algebra-supervised-learning}

\todo[inline]{Sistemare}
\subsection{Variabili aleatorie}\label{variabili-aleatorie}

\subsubsection{Bernoulli}\label{bernoulli}

Esito di un esperimento che può essere positivo o negativo.

\begin{verbatim}
P(X = i) = p   se i=1 
           1-p se i=0
\end{verbatim}

\subsubsection{Binomiale}\label{binomiale}

La probabilatà di avere \emph{i} successi su \emph{N} esperimenti è
uguale a

\begin{quote}
P(X=i) = (N su i)pi(1-p)N-i
\end{quote}

Il valore atteso di questa variabile è dato da \texttt{N*p} mentre la
varianza è \texttt{N*p*(1-p)}.

\subsubsection{Distribuzione uniforme}\label{distribuzione-uniforme}

Assume che in un intervallo \texttt{{[}a,b{]}} tutti i punti hanno la
stessa probabilità.

\begin{quote}
P(X = x) = 1 / (b-a) con \texttt{a\ \textless{}=\ x\ \textless{}=\ b}

P(X = x) = 0 altrimenti
\end{quote}

Il valore atteso di X (\texttt{E{[}X{]}}) è uguale a \texttt{(a+b)/2}

\subsubsection{Distribuzione normale
(Gaussaina)}\label{distribuzione-normale-gaussaina}

La distribuzione si concentra in un certo valore medio \texttt{mu} ed ha
la forma \emph{a campana}.

\begin{quote}
N(mu, sigma2)

%P(x) = {[}1 / sigma(√2Pi){]}*e(x-mu)\^{}2 / 2sigma\^{}2
\end{quote}

\subsection{Algebra lineare}\label{algebra-lineare}

\begin{quote}
M \euro{} Rm x d
\end{quote}

Somma di due matrici: le matrici A e B devono avere la stessa
dimensione, e la matrice somma ha come elementi la somma degli elementi
delle matrici.

\begin{quote}
C = {[}A + B{]}i,j = {[}a{]}i,j + {[}b{]}i,j
\end{quote}

Per fare il prodotto di due matrici è necessario che siano di dimensioni
compatibili.

\begin{quote}
A \euro{} Rm x d B \euro{} Rd x k L'emento (i,j) della matrice C = A * B
è uguale alla somma del prodotto riga i-esima di a e colonna j-esima di
B
\end{quote}

La matrice trasposta di una matriche è la stessa matrice
``\emph{ribaltata}'' sulla diagonale.

\begin{quote}
(AB)T = BTAT
\end{quote}

Un vettore è una matrice di una sola colonna.

Il prodotto scalare tra due vettori è la sommatoria del prodotti dei
vari elementi del prodotto.

Due vettori si dicono ortogonali quando il loro prodotto scalare è 0.

Due vettori si dicono correlati se il loro prodotto scalare è maggiore
di 0, in caso contrario si dicono scorrelati.

La lunghezza di un vettore (norma2, distanza eculidea) è definita come
la radice quadrata della sommatoria dei vari elementi del vettore,
eleveati al quadrato.

Allo stesso modo il quadrato della lunghezza è la sommatoria dei
quadrati degli elementi del vettore.

Il prodotto scalare tra due vettori è anche uguale al prodotto delle
lunghezza dei due vettori, moltiplicato anche per il coseno dell'angolo
tra i due vettori.

La distanza tra due vettori è la norma della differenza tra i due
vettori.

Matrice inversa e determinante.

Utilizzando le matrici è possibile risolvere i sistemi lineari.

Una matrice pseudo inversa è un qualcosa di simile ad una matrice
inversa per le matrici rettangolari.

\begin{quote}
A+ = AT(AAT)-1
\end{quote}

\subsubsection{Autovalori e autovettori}\label{autovalori-e-autovettori}

\begin{quote}
A * e = lambda * e A matrice e vettore
\end{quote}

\texttt{e} è un autovettore della matrice A e \texttt{lambda} è il
corrispondente autovalore.

\textbf{Traccia}: la traccia di una matrice è la somma degli elementi
nella diagonale.

Una matrice si dice \textbf{simmetrica} se tutti gli autovalori sono
maggiori di 0.

\subsection{Supervised Learning}\label{supervised-learning}

Si vuole tradurre un insieme di dati in ingresso \emph{X} in un insieme di dati di uscita \emph{Y}.

Anche in questo caso c'è un \emph{oracolo} che funziona in modo stocastico e che sceglie un oggetto \emph{x} in \emph{X} secondo una certa probabilità \emph{P(x)} e sceglie \emph{y} in \emph{Y} in base a \emph{P(y\textbar{}x)}.

L'obiettivo che si vuole raggiungere è quello di approssimare queste
probabilità.

Cosa importante, questo oracolo non sempre è una funzione, questo perché
può capitare che ad uno stesso \emph{x} corrispondano \emph{y} diversi.

\subsubsection{Operativamente}\label{operativamente}

Si dispone di una serie di coppie \emph{(x,y)} che seguono lo schema naturale, l'insieme di queste coppie prende il nome di \textbf{training set}.

Viene quindi scelta un funzione \emph{h} che prende il nome di
\textbf{ipotesi}, definita nello spazio delle ipotesi \emph{H} tale che,
da valori presenti nell'insieme \emph{X}, restituisca dei valori
nell'insieme \emph{Y}.

L'apprendimento consiste quindi nell'andare a scegliere l'\emph{h}
migliore in modo che approssimi bene i dati presenti nel training set e
che riesca a generalizzare e predirre i corretti valori \emph{y} anche
per valori di \emph{x} non presenti nel training set.

Da ciò segue che possono essere commessi due tipi di errori:

\begin{itemize}
\item
  \textbf{Errore empirico}: è l'errore commesso da \emph{h} in media,
  all'interno del training set. In altre parole è l'errore medio
  dell'ipotesi sul training set.
  $$
  error_{Tr}(h) = \#\{(x,y) \in Tr \: | \: h(x) \neq y\}
  $$
\item
  \textbf{Errore ideale}: è l'errore commesso da \emph{h}~su una
  qualsiasi coppia \emph{(x,y) \textasciitilde{} P(x,y)}, come media su
  un'insieme infinito di coppie. Questo errore può essere solamente
  stimato.
  $$
  error_D(h) = P_{(x,y) \in D}(h(x) \neq y)
  $$
\end{itemize}

Per calcolare una stima dell'errore ideale si può usare un \textbf{test
set}, cioè un altro insieme di coppie \emph{(x,y)} che non compaiono nel
training set. Questa discriminazione è importante perché, se così non
fosse, l'errore ideale sarebbe influenzato dall'errore empirico.

\emph{\textbf{Riassumendo}: l'errore empirico è quello che si fa sui dati che si
conoscono, l'errore ideale è quello che si fa su dei dati nuovi.}

Dal momento che lo spazio delle ipotesi non può coincidere con tutte le
funzioni calcolabili è necessario fare delle assunzioni sulla funzione
oracolo. 
Queste assunzioni prendono il nome di \textbf{bias induttivo} e
derivano da delle conoscenze a priori che abbiamo sul dominio e
vengono utilizzate per fare delle previsioni induttive sui dati.

Fanno parte del bias induttivo:

\begin{itemize}
\item
  Come vengono rappresentati gli esempi;
\item
  Come viene modellato lo spazio delle ipotesi \emph{H};
\item
  La funzione obiettivo per la ricerca nello spazio \emph{H}, cioè come
  viene scelta la funzione \emph{h}.
\end{itemize}

\subsubsection{Regressione polinomiale}\label{es-regressione-polinomiale}

Si vuole trovare una funzione polinomiale in grado di approssimare dei punti in un piano.

$$
TRAIN = \{(x_1,y_1),\ldots{},(x_n,y_n)\}
$$

In questo caso il bias induttivo è assumere che esista una funzione
polinomiale in grado di approssimare i vari punti.

Lo spazio delle ipotesi diventa quindi l'insieme dei vari polinomi e
l'apprendimento viene fatto sui vari coefficenti.

Dobbiamo quindi scegliere tra questo spazio un grado \emph{p} che va a
limitare i possibili polinomi (definzione di \emph{H}) e i vari
parametri della curva (ricerca nello spazio \emph{H}).
