% !TEX encoding = UTF-8
% !TEX TS-program = pdflatex
% !TEX root = ../apprendimento_automatico.tex
% !TEX spellcheck = it-IT
\section{Lezione 8 - Alberi di decisione 2}\label{lezione-8---alberi-di-decisione-2}

\subsection{Dove'è il bias induttivo degli alberi di decisione?}\label{doveuxe8-il-bias-induttivo-degli-alberi-di-decisione}

Con \textbf{candidate elimination} c'era l'incompletezza delle ipotesi
ma la ricerca all'interno dello spazio è esaustiva, mentre negli alberi
di decisione, c'è la completezza per quanto riguarda lo spazio delle
ipotesi ma la ricerca non è completa in quanto vengono effettuate scelte
greedy.

Un altro bias induttivo è che tutti gli attributi che producono un
guadagno entropico alto si trovano vicino alla radice.

\subsection{Casi speciali}\label{casi-speciali}

\subsubsection{Attributi continui}\label{attributi-continui}

Uno o più attributi hanno dei valori continui, escluso il target che
rimane binario o con un numero discreto di possibili valori.

La soluzione è quella di trasformare dinamicamente un attributo continuo
\emph{A} nell'attributo booleano $A_c$ in modo che sia \texttt{true} se il
valore di \emph{A} è minore di una certa soglia \emph{c}.

Il tutto sta ne scegliere la soglia \emph{c} migliore cioè che
corrisponde al massimo guadagno entropico.

Si è dimostrato che il valore ottimo di soglia si localizza nel valore
di mezzo tra due valori a cui corrisponde un target diverso.

Da notare che con ID3 un attributo può essere utilizzato soltanto una
volta, in questo caso però è possibile riutilizzare l'attributo con un
\emph{c} diverso.

\subsubsection{Attributi con costi}\label{attributi-con-costi}

In alcune situazioni andare a verificare il valore assunto da un
attributo potrebbe avere un costo.

Può essere preferibile quindi testare prima gli attributi meno costosi,
serve quindi un criterio per la selezione dell'attributo ottimo che
tiene conto dei costi.

Alcuni criteri sono:

\begin{itemize}
\item \textbf{Diagnosi medica} $\frac{2^{Guadagno(S,A)}-1}{(Costo(A)+1)^w}$ con
\emph{w} tra 0 e 1 (più vicino a 1 è \emph{w} più peso si da al costo).

\item \textbf{Percezione robotica}: $\frac{Guadagno^2(S,A))}{Costo(A)}$
\end{itemize}

\subsubsection{Attributi con valori mancanti}\label{attributi-con-valori-mancanti}

In alcuni casi si vuole classificare qualcosa che non ha tutti i dati
per gli attributi.

Questi casi possono essere trattati in vari modi diversi:

\begin{itemize}
\item
  Utilizzare per \emph{A} il valore più comune nell'insieme degli esempi
  associato al nodo interno.
\item
  Come prima, solo che vengono considerati solamente esempi con target
  uguale a quello dell'esempio corrente (ovviamente devo sapere il
  valore del target).
\item
  Considerare tutti i valori $a_i$ che può assumere l'attributo e la
  loro probabilità di occorrenza nell'insieme degli esempi associati al
  nodo interno e andare sostituire l'esempio corrente \emph{(x,target)}
  con delle istanze frazionarie per ogni possibile valore di \emph{A},
  ognuna con un peso pari alla probabilità. Quando devo scoprire il
  target di un esempio si provano tutti i possibili valori e poi si fa la media pesata dei valori ottenuti, fornendo in risposta la classe più probabile.
\end{itemize}

\subsection{Overfitting}\label{overfitting}

Cioè l'ipotesi è molto accurata sui valori di training, ma sui valori di
test risulta meno accurata.

All'aumentare della complessità dell'albero creato, l'accuratezza
dell'albero sui dati di training aumenta, ma una volta provata con i
dati di test, l'accuratezza cala drasticamente.

Ciò può essere causato dalla presenza di rumore nei dati o dal numero ridotto
di esempi disponibili per l'addestramento.

Si è osservato che fino ad un certo livello di complessità l'accuratezza
in training è molto simile all'accuratezza in test, è quindi importante
\textbf{potare} gli alberi complessi.

Ci sono però due problemi:

\begin{enumerate}
\item
  Come si effettua la potatura?
\item
  Quando fermarsi con la potatura o con l'apprendimento?
\end{enumerate}

Per quanto riguarda il problema (2) ci sono varie soluzioni:

\begin{itemize}
\item
  Valutare le prestazioni sull'insieme di apprendimento usando un test
  statistico;
\item
  Valutare le prestazioni su un'insieme separato di validazione;
\item
  Usare un principio di \textbf{minimizzazione della lunghezza di
  descrizione (MDL)}: \emph{min\_Tree{[}size(tree) -
  size(errori(tree)){]}.}
\end{itemize}

\subsubsection{Come potare}\label{come-potare}

\paragraph{Reduce error pruning}\label{reduce-error-pruning}

Effettuare il pruning di un nodo consiste nel rimuovere il sotto albero
che ha origine in quel nodo, trasformando il nodo potato in una foglia e
assegnandogli come valore la classificazione più comune tra gli esempi
di training associati a quel nodo.

I nodi vengono rimossi solamente se le prestazioni dell'albero potato
non peggiorano rispetto la versione originale, confrontate sul
validation set.

\begin{itemize}
\item
  Dividere il training set in due sottoinsiemi, uno per fare training e
  l'altro per fare validazione.
\item
  Ripetere fino a quando le prestazioni peggiorano:
  \begin{itemize}
  \item
    Per ogni nodo interno \emph{n} valutare l'impatto del nodo sul
    sottoinsieme di valutazione avendo potato il nodo
  \item
    Effettuare la potatura che porta alle prestazioni migliori
    sull'insieme di valutazione.
  \end{itemize}
\end{itemize}

Al sotto albero radicato in \emph{n} si sostituisce la foglia con
etichetta uguale alla classe più frequente nell'insieme degli esempi
associati al nodo \emph{n}.

\paragraph{Rule-Post pruning}\label{rule-post-pruning}

\begin{itemize}
\item
  Si genera una regola $R_i$ per ogni cammino \emph{path(r, fi)}
  dalla radice \emph{r} alla foglia \emph{i}-esima $f_i$.
\item
  Si effettua la potatura indipendentemente su ogni regola $R_i$:

  \begin{itemize}
  \item
    Si stimano le prestazioni utilizzando solo $R_i$ come
    classificatore
  \item
    Si rimuovo le precondizioni $ (Attr_{i} = v_{i_k})$ (una o più) che conducono ad un aumento
    della stima delle prestazioni utilizzando un approccio greedy.
  \end{itemize}
\item
  Si ordinano le $R_i$ potate per ordine crescente di prestazione
  (evitando i conflitti)
\item
  Eventualmente si aggiunge come classificazione di default la classe più
  frequente
\end{itemize}

$R_i$ è del tipo:

$$
\text{IF } (Attr_{i_1} = v_{i_1}) \wedge \ldots \wedge (Attr_{i_k} = v_{i_1}) \text{ THEN } label_{f_i}
$$

La classificazione di una nuova istanza a partire delle regole
ordinate avviene seguendo l'ordine stabilito per le regole:

\begin{itemize}
\item
  La prima regola la cui precondizione è soddisfatta dalla istanza è
  usata per generare la classificazione
\item
  Se nessuna regola ha le condizioni soddisfatte, si utilizza la regola
  di default per la classificazione, cioè si ritorna la classi più
  frequente nell'insieme di apprendimento.
\end{itemize}
