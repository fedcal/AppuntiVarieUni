% !TEX encoding = UTF-8
% !TEX program = pdflatex
% !TEX root = AALP.tex
% !TEX spellcheck = it-IT

% 20 Ottobre 2016
% \section sistemi di tipi
% \subsubsection Teorema di Preservazione (Subject reduction)

% Dimostrazione del lemma di sostituzione (spostata al posto corretto in lezione5)

\subsubsection{Corollario di Subject-reduction}

\begin{center}
	Se $\emptyset \vdash M : T$ e $M \rightarrow^* M'$, allora $\emptyset \vdash M' : T$
\end{center}

\paragraph{Dimostrazione}

La dimostrazione viene fatta per induzione sul numero di passi che da $M$ portano ad $M'$.

Il caso base è quello con il cammino di lunghezza 0, ovvero quando $M = M'$ e la tesi è trivialmente vera.

Nel caso induttivo ho che $M \rightarrow^{*}_k M_1 \rightarrow M'$, pertanto per ipotesi induttiva ho che $\emptyset \vdash M_1 : T$. 

Ho poi che $M_1 \rightarrow M'$ e quindi posso applicare subject-reduction per ottenere che $\emptyset \vdash M' : T$.

\subsubsection{Teorema di Safety - Completo}

\begin{center}
	Se $\emptyset \vdash M : T $ e $M \rightarrow^* M'$ e $M' \not\rightarrow$, allora $M'$ è un valore.
\end{center}

\paragraph{Dimostrazione}

Alle due ipotesi posso applicare il corollario di subject-reduction, ottenendo che $\emptyset \vdash M' : T$.

Per il teorema di progressione ho poi che $M'$ è un valore oppure $M'$ fa un passo e va in $M''$, ma per ipotesi $M'$ non fa un passo e quindi $M'$ è per forza un valore.

\subsection{Esercizio sul typing}

Da notare che nel contesto iniziale non servono giudizi di tipo perché tutte le variabili che compaiono sono all'interno delle funzioni e che, non essendoci informazioni sui tipi, questi sono ignoti e pertanto è necessario fare anche inferenza durante la risoluzione dell'albero.

\begin{prooftree}	
	
	% App-1
	\AxiomC{$\checkmark$ $\textbf{\textit{T}} = \textbf{\textit{T}}_1 \rightarrow \textit{\textbf{T}}_3$}
	\LeftLabel{(\textsc{Var})}
	\UnaryInfC{$y : T, x : T' \vdash y : T_1 \rightarrow T_3$}
	
	% App-2
	\AxiomC{*}
	\LeftLabel{\textsc{(If)}}
	\UnaryInfC{$y : \textbf{\textit{T}}_1\rightarrow \textbf{\textit{T}}_3, x:T' \vdash \text{if true then }x \text{ else }y \: x : T_1$}
	\LeftLabel{\textsc{(App)}}
	\BinaryInfC{$y : T, x: T' \vdash y \: (\text{if true then }x \text{ else }y \: x) : \: T_3$}
	
	\LeftLabel{\textsc{(Fun)}}
	\UnaryInfC{$y : T \vdash \fn x : \textbf{\textit{T'}} . (y \: (\text{if true then }x \text{ else }y \: x)) : \textbf{\textit{T}}_2 = \textbf{\textit{T'}} \rightarrow \textbf{\textit{T}}_3$}
	
	\LeftLabel{\textsc{(Fun)}}
	\UnaryInfC{$\emptyset \vdash \fn y : \textbf{\textit{T}} . \fn x : T' . (y \: (\text{if true then }x \text{ else }x \: y)) : \textbf{\textit{T}} \rightarrow \textbf{\textit{T}}_2$} % T' \rightarrow ?
\end{prooftree}

\noindent Dal ramo sinistro di \textsc{(App)} riesco a chiudere con $T = T_1 \rightarrow T_3$ e quindi riporto l'informazione anche sul ramo destro.
La derivazione continua applicando \textsc{(IfThenElse)} e con $\Gamma = y:T_1 \rightarrow T_3, x:T'$:

\begin{prooftree}
	%if 1
	\AxiomC{$\checkmark$}
	\LeftLabel{\textsc{(True)}}
	\UnaryInfC{$\Gamma \vdash \true : \Bool$}
	
	%if 2
	\AxiomC{$\checkmark$  $\textbf{\textit{T}}_1 = \textbf{\textit{T'}}$}
	\LeftLabel{\textsc{(Var)}}
	\UnaryInfC{$y: T_1 \rightarrow T_3, x:T' \vdash x : T_1$}
	
	%if 3
	\AxiomC{**}
	\LeftLabel{\textsc{(App)}}
	\UnaryInfC{$y : \textbf{\textit{T'}} \rightarrow T_3, x:T' \vdash y \: x : \textbf{\textit{T'}}$}
	% per quello che ho scoperto 
	\UnaryInfC{$y : T_1 \rightarrow T_3, x:T' \vdash y \: x : T_1$}
	\LeftLabel{\textsc{(If)}}
	\TrinaryInfC{(*)$\Gamma \vdash \text{if true then }x \text{ else }y \: x : T_1$}
\end{prooftree}

\noindent Dal ramo centrale della regola \textsc{(IfThenElse)} scopro che $T_1 = T'$ e quindi aggiorno il ramo destro e il contesto, che ora è  $\Gamma = y:T' \rightarrow T_3, x:T'$.

\begin{prooftree}
	\AxiomC{$\checkmark$}
	\LeftLabel{\textsc{(Var)}}
	\UnaryInfC{$y:T' \rightarrow T_3 , x:T' \vdash y : T_4 \rightarrow T'$}
	
	\AxiomC{$\checkmark$}
	\LeftLabel{\textsc{(Var)}}
	\UnaryInfC{$y : T' \rightarrow T_3 , x :T'\vdash x : T_4$}
	
	\LeftLabel{\textsc{(App)}}
	\BinaryInfC{(**)$\Gamma \vdash y \: x : T'$}
\end{prooftree}

\noindent Da quest'ultimo albero trovo che $T' = T_4 = T_3$. Andando a sostituire il tutto trovo che:

\begin{itemize}
	\item $y : T' \rightarrow T'$
	\item $x : T'$
	\item $T_2 = T' \rightarrow T_3 = T' \rightarrow T'$
	\item $T = T_1 \rightarrow T_3 = T' \rightarrow T'$
	\item Il tipo del programma è quindi $(T' \rightarrow T') \rightarrow (T' \rightarrow T')$
\end{itemize}

\chapter{Estensioni del linguaggio funzionale}

\section{Unit}

Nei linguaggi funzionali, ogni funzione deve ritornare sempre un valore, tuttavia può capitare che delle funzioni non debbano ritornare un valore. In questo caso viene ritornato il tipo \text{Unit} che può assumere l'\textbf{unico} \textbf{valore} \texttt{()} o \text{unit}.

Ad esempio la funzione \texttt{print} in Scala ritorna un valore di questo tipo:

\begin{lstlisting}[language=Scala]
def print( x: Any) : Unit = {...}
\end{lstlisting}

\noindent allo stesso modo anche l'assegnamento in Scala ritorna un valore \text{Unit} mentre in C/C++ viene solitamente ritornato il valore assegnato (per concatenare le operazioni di assegnamento).

Aggiungere \text{Unit} al linguaggio vuol dire estendere la definizione:

\begin{align*}
	M &::= \ldots \: | \: \text{unit} \: | \: \ldots \\
	v &::= \ldots \: | \: \text{unit} \: | \: \ldots \\
	T &::= \ldots \: | \: \text{Unit} \: | \: \ldots 
\end{align*}

\noindent e allo stesso modo serve una regola di tipo:

\begin{prooftree}
	\AxiomC{}
	\LL{Unit}
	\UnaryInfC{$\Gamma \vdash \text{unit : Unit}$}
\end{prooftree}

\noindent La killer-feature del tipo \text{Unit} è la possibilità di implementare il call-by-name utilizzando il call-by-value.

Ad esempio, tornano alle nostre asserzioni in Scala

\begin{lstlisting}[language=Scala, caption=Version ``standard'' delle asserzioni]
var assEnabled = true
...
def assert(pred : Bool) = 
	if (assEnabled && !pred)
		throw Exc
...

assert(saldoConto() > 0)
\end{lstlisting}


\noindent possiamo utilizzare \text{Unit} per far funzionare il codice precedente come se fosse call-by-name ma senza specificarlo:

\begin{lstlisting}[language=Scala, caption=Version ``standard'' delle asserzioni]
var assEnabled = true
...
def assert(pred : Unit => Bool) = 
	if (assEnabled && pred() == false) /** l'invocazione con () passa Unit*/
		throw Exc
...
assert(fn x:Unit . (saldoConto() > 0) ) /** pseudo scala */
assert( () =>  (saldoConto() > 0) ) /** sintassi corretta (la funzione anonima non ha parametri) */
\end{lstlisting}

\noindent In questo caso con la sintassi call-by-value viene calcolato il valore del parametro, ma in questo caso il parametro è una funzione che è già un valore.
Quindi, se le asserzioni sono disabilitate, alla chiamata della funzione \texttt{assert} non viene chiamata la funzione \texttt{saldoConto()} perché la funzione è già un valore.

Per riportare la stessa cosa nel nostro linguaggio funzionale con il call-by-value:

$$
\fn x:T.M \: N \rightsquigarrow \: \big(\fn y : \text{Unit} \rightarrow T . M\{x := y \: \text{unit}\} \big) \:\:  \big(\fn z : \text{Unit} . N \big)
$$

\subsection{Implementazione del while}

Vogliamo definire una funzione che si comporta come il while in Scala.

\begin{lstlisting}[language=Scala, caption=I parametri devono essere dichiarati come call-by-name oppure  ]
def WHILE(cond : =>Bool , command : =>Unit ) : Unit = 
	if (cond == false ) () /** In scala il return è opzionale, () è il valore di Unit*/
	else {
		command
		WHILE(cond, command)
	}

/** Uso della funzione */
var a = 0
WHILE(a < 4 , {print(a); a=a+1})
\end{lstlisting}

\noindent Dato che sono in call-by-value i parametri vengono valutati prima di invocare per la prima volta il while, quindi o li passo per by-name oppure li passo utilizzando una funzione. 
Questo perché sia il codice del corpo che la condizione devono poi essere rieseguiti (valutati) nelle successive iterazioni. 


