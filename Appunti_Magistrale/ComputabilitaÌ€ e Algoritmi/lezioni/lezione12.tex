% !TEX encoding = UTF-8
% !TEX TS-program = pdflatex
% !TEX root = computabilità e algoritmi.tex
% !TEX spellcheck = it-IT
\paragraph{Soluzione}\label{soluzione-esercizio}

La funzione \emph{f(x)} viene definita per casi e dal momento che i vari
predicati sono decidibili ed esaustivi, quindi almeno uno dei casi è
vero.

$$
f(\vec{x}) = f_1(\vec{x}))\cdot \mathcal{X}_{Q_1} + \ldots +	 f_m(\vec{x}))\cdot \mathcal{X}_{Q_m}
$$

Quando un caso è vero, viene calcolata la funzione associata, che è
totale. Se questa funzione non fosse totale, potrebbe essere che il
predicato vero su un certo \emph{x}, ma che la funzione su
quell'\emph{x} non sia definita e quindi neanche \emph{f(x)}
risulterebbe definita, perdendo così la totalità.

La calcolabilità deriva dal fatto che la definizione per casi viene
fatta con una serie di \emph{if} che è dimostrato essere calcolabile e
le funzioni $\mathcal{X}_{Q_i}$ sono calcolabili, perché i predicati sono
decidibili.

\subsubsection{Algebra della decidibilità}\label{algebra-della-decibilituxe0}

I predicati decidibili sono chiusi rispetto negazione, congiunzione e
disgiunzione.

Ovvero se \textit{Q} e \textit{Q'} sono due predicati in $\mathbb{N}^k$ decidibili allora anche:

\begin{enumerate}
\item $\neg Q(\vec(x))$
\item $ Q(\vec{x}) \wedge Q'(\vec{x})$
\item $ Q(\vec{x}) \vee Q'(\vec{x}) $
\end{enumerate}

Questo perché le relative funzioni che li calcolano possono essere definite come:

\begin{enumerate}
	\item $ \mathcal{X_{\neg Q}}(\vec{x}) = \overline{sg}( \mathcal{X_{Q}}(\vec{x})) $
	\item $ \mathcal{X_{Q \wedge Q'}}(\vec{x})= \mathcal{X_{Q}}(\vec{x}) \cdot \mathcal{X_{Q'}}(\vec{x})$
	\item $ \mathcal{X_{Q \vee Q'}}(\vec{x})= sg(\mathcal{X_{Q}}(\vec{x}) + \mathcal{X_{Q'}}(\vec{x}))$
\end{enumerate}

Tutte le funzioni così definite sono calcolabili perché ottenute da composizioni di funzioni calcolabili.

\subsubsection{Somma e prodotto dei valori di una funzione}

Data una funzione totale e $ f : \mathbb{N}^{k+1} \rightarrow  \mathbb{N}$ totale e calcolabile, è possibile definire le due funzioni che effettuano la somma e il prodotto dei primi \textit{y} valori della funzione.

\begin{align*}
	s(\vec{x}, y) &= \sum_{z < y} f(\vec{x},z) \\
	p(\vec{x}, y) &= \prod_{z < y} f(\vec{x},z) 
\end{align*}

Entrambe le funzioni sono calcolabili e totali perché possono essere definite induttivamente (ricorsivamente) a partire da delle funzioni calcolabili.

\begin{align*}
s(\vec{x}, y) &= \begin{cases} \sum_{z < 0}f(\vec{x},z) = 0, &\text{ se $ y = 0 $} \\
\sum_{z < y+1}f(\vec{x},z) = \sum_{z < y}f(\vec{x},z) + f(\vec{x},z), &\text{ altrimenti}
\end{cases} \\
p(\vec{x}, y) &=  \begin{cases} \prod_{z < 0}f(\vec{x},z) = 1, &\text{ se $ y = 0 $} \\
\prod_{z < y+1}f(\vec{x},z) = \prod_{z < y}f(\vec{x},z) \cdot f(\vec{x},z) &\text{ altrimenti}
\end{cases}
\end{align*}

La totalità deriva dal fatto che \textit{f} è totale.

\subsubsection{Quantificazione limitata}

Combinando algebra della decidibilità e quanto detto nel paragrafo precedente è possibile la decidibilità di $ \forall $ e $ \exists $.

Dato un predicato $ Q(\vec{x},z) $ per calcolare se $ \forall z < y, Q(\vec{x},z) $ è possibile utilizzare la funzione:

$$
\mathcal{X}_{Q_\forall} = \prod_{z < y} \mathcal{X}_Q(\vec{x},z)
$$

In modo simile è possibile calcolare $ \exists z < y, Q(\vec{x},z) $:

$$
\mathcal{X}_{Q_\exists} = sg(\sum_{z < y} \mathcal{X}_Q(\vec{x},z))
$$

Trattandosi della composizione di funzioni calcolabili e totali, le funzioni così ottenute sono a loro volta calcolabili e totali.

\section{Minimalizzazione limitata}

Data una funzione $ f(\vec{x},z) : \mathbb{N}^{k+1} \rightarrow \mathbb{N}$ calcolabile e totale è possibile definire una funzione 

$$
h(\vec{x},y) = \mu z< y | f(\vec{x},z) = 0
$$

$ h $ è ancora una funzione $ \mathbb{N}^{k+1} \rightarrow \mathbb{N} $ e viene calcolata come il minimo valore di \textit{z} minore di \textit{y} e tale che $ f(\vec{x},z) $ sia uguale a 0 (tipicamente l'uguale a 0 viene omesso). 

La definizione più precisa è:

$$
h(\vec{x}, y) = \mu z < y . f(\vec{x},z) = \begin{cases}
\text{minimo $ z < y $ tale che $f(\vec{x},z) = 0$ se questo esiste} \\
y, \text{altrimenti}
\end{cases}
$$

Per come è definita, questa funzione risulta essere \textbf{totale} e \textbf{calcolabile}.
Intuitivamente è calcolabile perché si tratta di calcolare \textit{f} per vari valori, serve però una dimostrazione più formale, fatta per ricorsione primitiva.

\begin{align*}
	h(\vec{x}, 0) &= 0\\
	h(\vec{x}, y+1) &= 	\begin{cases}
										h(\vec{x},y) < y, &\text{\textit{f} si annulla su un valore minore di \textit{y}, viene resituito $ h(\vec{x},y) $}\\
										h(\vec{x},y) = y, &\text{per tutti i valori di minori \textit{y} non c'è uno 0 } \begin{cases}
										\text{se } f(\vec{x},y) = 0 \rightarrow y \\ 
										\text{se } f(\vec{x},y) \neq 0 \rightarrow y+1
										\end{cases}
										\end{cases}
\end{align*}

La seconda parte può essere facilmente tradotta nell'espressione

$$
(y \dotminus h(\vec{x},y)) \cdot (h(\vec{x},y)) + \overline{sg}(y \dotminus h(\vec{x},y)) \cdot (y + sg(f(\vec{x},y)))
$$

Dal momento che la funzione \textit{h} può essere definita per ricorsione primitiva e per composizione di funzioni calcolabili, anche lei è calcolabile.

\subsection{Funzioni calcolabili per ricorsione limitata}

Utilizzando la ricorsione limitata è possibile dimostrare la calcolabilità di varie funzioni.

\subsubsection{Numero di divisori di $x$}

\begin{align*}
	D(x) &= \text{\# divisori di } x \\
			&= \sum_{y \leq x}(\overline{sg}(rm(y,x))) 
\end{align*}

Dove \textit{rm} è la funzione resto, precedentemente dimostrata calcolabile.

\subsubsection{Numeri primi}

Dimostrare la calcolabilità dei funzioni che lavorano con i numeri primi è importante perché torneranno utili in futuro.

\begin{align*}
	Pr(x) &= \text{``$x$ è primo ''} \\
			 &= \overline{sg}(|D(x) - 2|)
\end{align*}

\begin{align*}
	P_x &= \text{``$x$-esimo numero primo, per convezione: ''} P_0 = 0, P_1 = 2, \ldots \\
	&= \begin{cases}
	P_0 = 0& \\
	P_{x+1} = \mu z \leq (P_x! + 1) \: | Pr(z) \cdot \underbrace{\overline{sg}(P_x +1 \dotminus z)}_{1 \text{ se } z > P_x  } \dotminus 1| &
	\end{cases}
\end{align*}

\begin{align*}
	(x)_y 	&= \text{esponente di  } P_y \text{ nella decomposizione di } y \\
			   &= \text{max } z \: P_{y}^z \text{ divide } x \rightarrow \mu z \leq x \text{ tale che } P_{y}^{z+1} \text{ non divide } x \\
			   &= \mu z \leq x \: \overline{sg}(rm(P_{y}^{z+1},x))
\end{align*}

\subsubsection{Esercizio - mcm, MCD, radice di x}

Dimostrare che sono calcolabili:

\begin{itemize}
	\item $ floor(\sqrt{x}) $
	\item $mcm(x,y))$
	\item $MCD(x,y)$
\end{itemize}


\section{Codifica di coppie}

La funzione di \textit{fibonacci} non può essere definita per ricorsione primitiva, perché il passo induttivo richiede una coppia di valori precedenti.

\`{E} però possibile definire una funzione $\prod : \mathbb{N} \times \mathbb{N} \rightarrow \mathbb{N}$ che codifica il valore di una coppia in un unico numero:

$$
\prod (x,y) = 2^x(2y+1) \dotminus 1
$$

Questa funzione risulta essere biunivoca perché è possibile definire l'inversa:

$$
\prod^{-1}(x) = ((n+1)_1, \frac{1}{2}(\frac{n+1}{(n+1)_1})-1)
$$

La funzione inversa è effettiva, perché è definita in termini di componenti calcolabili, anche se la definizione di funzione calcolabile non è stata vista per funzioni $\mathbb{N} \rightarrow \mathbb{N} \times \mathbb{N}$.

Utilizzando la funzione accoppiamento, è possibile definire la funzione \textit{fib} per ricorsione primitiva:

\begin{align*}
	g(x) &= \prod(fib(x), fib(x+1)) \\
	g(0) &= \prod(fib(0), fib(1)) = \prod(1,1) = 5 \\
	g(x+1) &= \prod(fib(x+1), fib(x+2)) \\
				 &= \prod(fib(x+1, fib(x)+fib(x+1)) \\
				 &= \prod(\prod_2(g(x)),\prod(\prod_1(g(x))+\prod_2(g(x)))\\		 
\end{align*}

Dove $\prod_1$ e $\prod_2$ sono rispettivamente le funzioni per il calcolo del primo e del secondo elemento della coppia.

Così facendo è stata dimostrata la calcolabilità di \textit{g} per ricorsione primitiva, ma $fib(x) = \prod_1(g(x))$ e di conseguenza \textit{fib} è calcolabile per composizione di funzioni calcolabili.

\section{Minimalizzazione illimtata}

Data $ f : \mathbb{N}^{k+1} \rightarrow \mathbb{N} $, si vuole definire $ h : \mathbb{N}^k \rightarrow \mathbb{N} $ tale che calcoli il minimo \textit{z} che azzera la funzione \textit{f}, ovvero:

$$
h(\vec{x}) = \mu z.f(\vec{x},z)
$$

Ci sono però dei problemi se $ f(\vec{x}, z) $ è sempre diversa da zero, perché in questo caso $ h $ è $ \uparrow $.

Un altro problema si ha se la funzione è indefinita per un valore $ z' $ minore dello $ z $ che azzera la funzione. Anche in questo caso si ha che $ h $ è $ \uparrow $.

$$
h(\vec{x}) = \mu z.f(\vec{x},z) = \begin{cases}
\text{minimo \textit{z} tale che } f(\vec{x},z) = 0 \text{ se esiste e se } \forall z' < z, f(\vec{x},z')\downarrow \neq 0 \\
\uparrow \text{ altrimenti}
\end{cases}
$$

Alternativamente, definendo $ Z_{f, \vec{x}} = \{z | f(\vec{x},z) = 0 \wedge \forall z' < z f(\vec{x},z') \downarrow \} $, si ha che \textit{h} è definita come

$$
h(\vec{x}) = \begin{cases}
\min Z_{f,\vec{x}} \text{ se } Z_{f,\vec{x}} \neq \emptyset \\
\uparrow \text{ altrimenti}
\end{cases}
$$

\subsection{Esercizi}

\subsubsection{Esercizio - Radice quadrata}
Dimostrare la calcolabilità di 

$$
f(x) = \begin{cases}
\sqrt{x} \text{ se \textit{x} è un quadrato} \\
\uparrow \text{ altrimenti}
\end{cases}
$$

\paragraph{Soluzione}

L'idea è quella di trovare un \textit{y} che elevato al quadrato è uguale a \textit{x}: $ y^2 - x = 0 $.

Si ha quindi che

$$ 
f(x) = \mu y.|y^2-x|
$$

ed è calcolabile perché minimizza illimitatamente una composizione di funzione calcolabile.

\subsubsection{Esercizio teorico}

Dimostrare che se $ f : \mathbb{N} \rightarrow \mathbb{N} $ è iniettiva, calcolabile e totale, anche la sua inversa è calcolabile.

\paragraph{Soluzione}

$$
f^{-1}(x) = \begin{cases}
y, &\text{ tale che } f(y) = x \\
\uparrow, &\text{ altrimenti}  
\end{cases}
$$

$$ 
f^{-1}(x) = \mu y.|f(y)-x|
$$

Perché l'uguaglianza sia rispettata è necessario che \textit{f} sia totale, dal momento che se per un certo \textit{y}, \textit{f} non è definita il programma che la calcola non termina, rendendo indefinita anche $ f^{-1} $.

C'è un barbatrucco per gestire anche la non totalità di \textit{f}, ovvero quello di eseguire contemporaneamente il calcolo per ogni \textit{y}, eseguendo tot passi alla volta per ognuno dei calcoli (\textit{dovrebbe essere dimostrato in futuro}).

Dal momento che \textit{f} è calcolabile si ha che anche l'inversa è calcolabile per minimizzazione illimitata di una composizione di funzioni calcolabili.

L'iniettività\footnote{Una funzione $ f: X\rightarrow Y $ si dice iniettiva se due elementi distinti del dominio hanno immagini distinte, ovvero $a_1\neq a_2$ implica $f(a_1)\neq f(a_2)$.} garantisce che il valore trovato sia quello corretto.


\subsubsection{Esercizio - Divisione}

$$
f(x,y) = \begin{cases}
\frac{x}{y}, &\text{ se } y \neq 0 \text{ e \textit{x} divisibile per \textit{y}}\\
\uparrow, &\text{altrimenti} 
\end{cases}
$$

\paragraph{Soluzione}

$$
f(x,y) = \mu k. |x \dotminus y\cdot k|
$$

C'è però un problema, perché la funzione così definita risulta calcolabile se \textit{x} e \textit{y} sono uguali a 0.

$$
f(x,y) = \mu k.(|x - y\cdot k| + \underbrace{\overline{sg}(y)}_{\text{vale 1 se \textit{y} è uguale a 0}})
$$

Così facendo se $ y=0 $ la minimalizzazione non converge.

Questo porta ad un discorso un po' più ampio sulla possibilità di aggiustare una funzione che si comporta quasi come un'altra funzione (§\ref{hotfix}).
