% !TEX encoding = UTF-8
% !TEX TS-program = pdflatex
% !TEX root = computabilità e algoritmi.tex
% !TEX spellcheck = it-IT

\section{Correzione compitino}

\subsection{Esercizio 1}

Definire la classe delle funzioni ricorsive e dimostrare che $ pow2(y)  \in \mathcal{PR} $

\subsubsection{Soluzione}

La definizione è negli appunti.

Per dimostrare che la funzione è primitiva ricorsiva la si definisce per ricorsione primitiva:

$$
\begin{cases}
pow2(0) &= 1 \\
pow2(y+1) &= double(pow2(y))
\end{cases}
$$

Serve definire anche $ double $ come primitiva ricorsiva:

$$
\begin{cases}
double(0) &= 0 \\
double(y+1) &= S(S(double(y)))
\end{cases}
$$

$ pow2 $ è quindi primitiva ricorsiva.

\subsection{Esercizio 2}

Analizzare la classe delle funzioni calcolabili $ \mathcal{C}' $ associate alla macchina URM' che al posto dell'istruzione di salto e dell'istruzione successore ha l'istruzione \texttt{JI(m,n,t)} che se trova i due registri uguali, prima incrementa il registro \texttt{m} e poi effettua il salto all'istruzione \texttt{t}.

\subsubsection{Soluzione}

Le istruzioni \texttt{J} e \texttt{S} della macchina URM possono essere replicate con 

\begin{lstlisting}[language=URM]
// J(m,n,t)
T(m,k)
JI(k,n,t)
// S(n)
Il: JI(n,n,l+1)
\end{lstlisting}

Serve poi la prova induttiva sul numero di istruzioni che è sempre quella, l'importante è specificare che nei passi intermedi della dimostrazione si ottiene un programma ibrido.

C'è poi da fare la dimostrazione dell'altro verso, che avviene in modo analogo.

\subsection{Esercizio 3}

Può esistere una funzione $ f $ totale e non calcolabile tale che $ cod(f) = \mathbb{P} $.

\subsubsection{Soluzione}

Perché sia totale basta che sia sempre definita, per ottenere la non calcolabilità la devo rendere diversa da tutte le funzioni calcolabili.

$$
f(x) = \begin{cases}
x, &\text{se $ x $ è pari}\\
2(\phi_{\frac{x-1}{2}}(x) +1), &\text{se $ x $ è disparie e $ \downarrow $} \\
0, &\text{ altrimenti}
\end{cases}
$$

Ma anche una definizione più semplice va bene:

$$
f(x) = \begin{cases}
2 (\phi_x(x) +1), &\text{ se }\phi_x { è totale} \\
0, \text{ altrimenti}
\end{cases}
$$

\chapter{Funzione universale}

Vogliamo un programma universale che prende in input il numero del programma e l'input da utilizzare e fornisce in output il risultato dell'esecuzione del programma sul dato input.

$$
\Psi_U(x,y) = \phi_{x}(y)
$$

La funzione universale ha in generale $ k+1 $ argomenti:

\begin{align*}
\Psi_U^{(k)} &: \mathbb{N}^{k+1} \rightarrow \mathbb{N} \\
\Psi_U(e, \vec{x}) = \phi_{e}^{k}(\vec{x})
\end{align*}

\section{Calcolabilità della funzione universale}

Per calcolare $ \Psi_U(e, \vec{x}) = \phi_{e}^{k}(\vec{x}) $ è necessario come prima cosa recuperare il programma rappresentato da $ e $ utilizzando $ \gamma^{-1}(e) = P_e$.

Dopodiché è necessario eseguire $ P_e(\vec{x}) $, la quale può produrre un output $\phi_{e}^{k}(\vec{x}) = \Psi_U(e, \vec{x}) $ oppure può non terminare e in questo caso $ \Psi_U(e, \vec{x})\uparrow $.

\subsection{Dimostrazione pseudo-formale}

\begin{verbatim}
	|r1|r2|r3|...|0|...
\end{verbatim}

$ C = \prod_{i \geq 1} = p_{i}^{r_i} $ con questa codifica della memoria è possibile accedere al contenuto del registro $ i $-esimo con $ r_i = (C)_i$.

Qualcosa sulla codifica dello stato del programma: s$ \sigma = \Pi(C,J) $.

$ C_k(e,\vec{x}, t)  = $ contenuto dei registri dopo $ t $ passi del programma $ P_e $ sull'input $ \vec{x} $.

$ J_k() = \begin{cases}
&\text{istruzione da eseguire dopo di passi di } P_e(\vec{x}) \text{, se non termina.}\\
&0 \text{ altrimenti}
\end{cases} $ 

C'è da dimostrare che $ C_k, J_k \in \mathcal{PR} \subseteq \mathcal{R}  = \mathcal{C}$, una volta fatto ciò è possibile definire:

$$
\Psi_U(e, \vec{x}) = \phi_{e}^{k}(\vec{x}) = \Big( C_k(e,\vec{x}, \mu t.J_k(e,\vec{x},t)) \Big)_1 \in \mathcal{R} = \mathcal{C}
$$

Sulla base di $ C_p(\vec{x},t) $ e $ J_p(\vec{x}, t) $ si possono definire per ricorsione primitiva:

$$
C_p()
$$


Servono prima un po' di funzioni ausiliarie.

Argomenti di un istruzione URM $i = \beta(\text{Istr}) $
\begin{align*}
	Z_{arg}(i) &= qt(4,i) +1 \\
	S_{arg}(i) &= qt(4,i) +1 \qquad \qquad i = (n-1)\times 4 \\
	T_{arg_1} &= \Pi_1 (qt(4,i)) +1 \qquad \qquad i =\ldots \\
	T_{arg_2} &= \Pi_2 (qt(4,i)) +1
\end{align*}

Effetto dell'esecuzione di un istruzione:

$$
zero(C,n) = qt(p_{n}^{(C)_n} , C)
$$

$$
succ(C,n) = p_n \cdot C
$$

$$
trasf(C,m,n) = zero(C,n) \cdot p_{n}^{(C)_m}
$$

Effetto dell'esecuzione di un istruzione $ i = \beta(\text{Itrs})$:

$$
change : \mathbb{N}^2 \rightarrow \mathbb{N}
$$

$$ 
change(C,i) = \begin{cases}
zero(C, Z_{arg}(i)) se \: rm(4,i) = 0 \\
succ(C, S_{arg}(i)) se \: rm(4,i) = 1 \\
trasf(C, T_{arg_1}(i), T_{arg_2}(i)) se rm(4,i) = 2 \\
C altrimenti (il salto non cambia la configurazione di memoria)
\end{cases}
$$ 


Prossima configurazione di memoria

\begin{align*}
nextconf(e, C, t) &= \text{Configurazione dei registri al prossimo passo partendo da } C \text{ e eseguendo la \textit{t}-esima istruzione di }P_e \\
	&= change(C, a(e,t))
\end{align*}

dove \textit{a} è la funzione che data la codifica di un programma ritorna il codice della $ t $-esima istruzione.

$ is(C,i,t) =$ numero dell'istruzione successiva se al passo corrente eseguo $ i = \beta(\text{Istr}) $ che si trova in posizione $ t $ del programma.

$$
is(C,i,t) = \begin{cases}
t+1, &\text{se } rm(4.i) \neq 3 \text{ oppure } (C)_{J_{arg_1}(i)} \neq (C)_{J_{arg_2}(i)}  \\
J_{arg_3}(i), &\text{ altrimenti}
\end{cases}
$$ 

$$
nextistr(e, C, t) = \text{prossima istruzione a partire da C ed eseguendo t}
$$

$$
nextistr(e,C,t) =\begin{cases}
is(C, a(e,t), t ), &\text{ se } is(C, a(e,t), t ) \neq l(e) +1 \\
0, &\text{ altrimenti}
\end{cases} 
$$

Siamo quasi alla fine

\begin{align*}
C_k (e, \vec{x}, 0) &= \prod\limits_{i=1}^{k} p_{i}^{r_i}\\
J_k(e,\vec{x}, 0) &= 1 \\
C_k (e, \vec{x}, t+1) &= nextconf(e, C_k(e,\vec{x}, t) , J_k(e, \vec{x}, t)) \\
J_k(e,\vec{x}, t+1) &= nextistr(e, C_k(e, \vec{x}, t) , J_k(e, \vec{x}, t)) 
\end{align*}

Dato che tutte le funzioni utilizzate, anche $ C_k $ e $ J_k $ sono primitive ricorsive e quindi calcolabili.

$$
\Psi_{U}^{(k)} = \Big(  C_k(e, \vec{x}, \mu t.J_k(e, \vec{x}, t)) \Big)_1 \in \mathcal{R}
$$



\section{Corollari-rio}

Sono decidibili i seguenti predicati

\begin{enumerate}
	\item $ H_k (e,\vec{x}, t) = P_e(\vec{x}) \downarrow $ in $ t $ o meno passi.
	\item $ S_k (e,\vec{x},y, t) = P_e(\vec{x}) \downarrow y $ in $ t $ o meno passi.
\end{enumerate}

Entrambi sono calcolabili perché basta fare andare il programma per $ t $ passi e osservare l'output.

$$
\mathcal{X}_{H_k}(e, \vec{x}, t) = \overline{sg}(J_k{e,\vec{x},t})
$$


$$
\mathcal{X}_{S_k}(e, \vec{x}, y, t) =  \overline{sg}((J_k(e,\vec{x},t)) + |\: y - (C_k(e,\vec{x},t) )_1\:|)
$$