% !TEX encoding = UTF-8
% !TEX program = pdflatex
% !TEX root = MEMOC.tex
% !TEX spellcheck = it-IT

\section{Recap sul branch and bound}

Termino quando non ci sono più nodi. In questo caso la soluzioni ottima è quella incumbent. Tipicamente però non si raggiunge la terminazione dell'algoritmo per questioni di tempo o memoria.

Se mi fermo prima della terminazione ho la garanzia che incumbent è almeno una soluzione feasible. Sappiamo inoltre che, dato che i valori rilassati sono numeri reali, la soluzione ottima per il problema intero è compresa tra il valore della soluzione incumbent e il più grande upper-bound.
Si può quindi definire un intervallo di ottimalità e fermarsi quando questo gap è inferiore di una certa soglia.
Anche se il gap è molto piccolo, non vuol dire che si è vicini alla soluzione ottima, infatti il branch and bound fa fatica a chiudere i gap piccoli.

Si parte quindi dal problema iniziale $P_0$ che è il problema originale senza i vincoli di integrità, lo si risolve e poi si aggiunge un nuovo vincolo del tipo $x_i \leq k_i$. Si ripete quindi questo procedimento fino a quando non si raggiunge la condizione di stop.

Ovviamente non conviene far ripartire da capo la risoluzione del problema lineare rilassato, ma è meglio utilizzare una strategia simile alla column-generation.
Tuttavia non possiamo far ripartire il simplesso dalla soluzione corrente $\overline{x}_{LR}$  (il pedice $LR$ sta per linear relaxation), perché il nuovo vincolo che è stato aggiunto è stato scelto appositamente per rendere infeasible la soluzione corrente.
Tuttavia, la soluzione del problema duale $\overline{u}$ rimane ottima e anche per il nuovo problema duale. Si può quindi utilizzare un metodo del \textit{dual-simplex} per ottenere una soluzione ottima per il nuovo problema primale in poche iterazioni.

(Nelle dispense c'è una grande parte che a noi non interessa)


\section{Polyheadral approch to LP}

\`E un altro modo per risolvere un problema di programmazione lineare intera (o mista).

Come riferimento utilizziamo il classico problema di minimizzazione in forma standard, con un sotto-insieme di variabili $x_i$ tali che $x_i \in \mathbb{Z}\: \forall i \in I$.

L'insieme dei vincoli e delle condizioni di integralità definiscono la feasibile-region del problema, una griglia di punti interi che sono racchiusi in un poliedro definito dal sistema $Ax = b, x \geq 0$.

Ma il poliedro $Ax=b, x \geq 0$ non è l'unico. Questo può essere visto a livello grafico: la nuova di punti della feasibile-region può essere racchiusa anche da un poliedro diverso da $Ax = b, x\geq 0$, che può essere genericamente definito come $A'x = b', x\geq0$.

Ricapitolando:
\begin{itemize}
	\item La feasible region è l'insieme dei punti feasibile per il problema LP.
	\item Una forumulation è un qualsiasi poliedro che racchiude tutti e soli i punti della feasible region. Per una stessa feasibile region è possibile trovare un numero infinito di formulation.
\end{itemize}  

Quindi in MILP ogni problema ha un numero infinito di possibili formulazioni e queste possono essere viste come i rilassamenti del branch and bound.
Pertanto sarebbe conveniente scegliere come rilassamento un poliedro che è il più possibile vicino alla soluzione intera ottima, perché poliedri (formulation) più piccoli danno bound migliori.
C'è da tenere conto che non sempre è possibile determinare se una formulation è più grande o più piccola dell'altra, perché può succedere che nessuna delle due contenga esattamente l'altra.

Più formalmente, la formulazione $\{Ax = b, x\geq0\}$ è \textbf{migliore} di un'altra formulazione $\{A'x = b', x\geq0\}$ per lo stesso problema quando:

$$
\underbrace{\{Ax = b, x\geq0\}}_{polyhedron} \subseteq \{A'x = b', x\geq0\}
$$

ovvero i punti di un poliedro sono tutti contenuti all'interno dell'altro poliedro. Ovviamente questo può essere determinato in modo algebrico.

Come problema d'esempio prendiamo \textbf{Uncapacitated Facility Location} (\textbf{UFL}): ci sono $n$ location $i = 1, \ldots, n$ dove possiamo costruire delle facilities. Ci sono $m$ utenti da soddisfare con le facilities $j = 1 \ldots m$. Ogni utente deve essere soddisfatto da solamente una facility. Servire un utente ha un costo $c_{ij}$. C'è anche un costo fisso $f_i$ da pagare se viene aperta una facility in location $i$.

La formulazione del problema è

\begin{itemize}
	\item $y_i = \begin{cases}
	1\quad&\text{apro una facility in }i \\
	0\quad&\text{altrimenti }i \\
	\end{cases}$
	\item $x_{ij} = \begin{cases}
	1\quad&\text{se la facility \textit{i} serve l'utente \textit{j}}\\
	0\quad&\text{altrimenti}
	\end{cases}$
	\item FO: $$
		\min \sum \limits_{i=1}^{n} f_i y_i + \sum\limits_{i,j} c_{ij}x_{i,j}
	$$
	\item ST: 
	\begin{itemize}
		\item $ \sum\limits_{i} x_{ij} = 1 \quad \forall j = 1 \ldots m$
		\item $x_{ij} \leq y_i \: \forall \: i,j$ oppure $\sum\limits_{j} x_{i,j} \leq M y_i  \:\forall i $
		\item $ 0 \leq x_{ij} \leq 1$, $ 0 \leq y_i \leq 1 \forall i,j$ 
		\item $ x_{ij}, y_i \in \mathbb{Z} $
	\end{itemize}
\end{itemize}

Le due formulazioni sono equivalenti, ma la prima richiede un maggior numero di vincoli. Come valore di $M$ per la seconda formulazione possiamo utilizzare il numero di utenti $m$, perché il massimo valore della sommatoria è $m$ ed avere un $M$ minimo porta ad avere dei buoni vincoli.

Ma quale delle due soluzioni conviene scegliere? Tipicamente meno vincoli ci sono, più è facile da risolvere il problema, ma avere tanti vincoli può portare ad avere un poliedro più piccolo.

Si può verificare che la prima formulazione è migliore, nel senso che definisce un poliedro più piccolo, perché se $(\overline{x},\overline{y})$ soddisfa la prima formulazione, questa soddisfa anche la seconda formulazione. 

Perché la coppia soddisfa tutti i vincoli del tipo
$$
\overline{x}_{ij} \forall\:i\forall\:j (*)
$$
e, fissato un $i$, la sommatoria di tutti quei vincoli su $j$ diventa:

$$
\sum\limits_{j} \overline{x}_{ij} \leq m \overline{y}_i
$$

ovvero $(\overline{x},\overline{y})$ soddisfa anche il vincolo della seconda formulazione.
Si può anche dimostrare che il contrario non vale.

Rimane comunque il dubbio su quale sia conveniente implementare, perché quando la formulazione geometricamente migliore è quella con più vincoli, c'è il rischio che il risolutore faccia comunque più fatica a trovarla. L'unica cosa che si può fare è provare.

A livello geometrico è comunque possibile definire la \textbf{formulazione ideale}, ovvero quella che è geometricamente migliore.
La soluzione ideale è quindi il \textbf{convex hull} della feasibile region, ovvero il minimo \textbf{convex set} che contenente la feasbile-region.
Un convex set $C$ è un insieme di punti, tali che $\forall x,y \in C$, il segmento che li congiunge è completamente contenuto in $C$.
Tutti i poliedri associati ad una formulation sono sempre dei convex set. 

Il \textbf{Fundamental theorem of Integer Linear Programming} dice che dato un (M)ILP (sotto forma di massimo), tale che la matrice $A$ contenga solo valori razionali\footnote{un calcolatore non riesce ad implementare i numeri razionali.}, il convex hull della feasible region è un poliedro, ovvero può essere espresso come $\{Cx\leq d, x\geq0 \}$. Non dimostriamo questo teorema.

Quindi dato un (M)ILP di massimo che definisce una feasbile region $X$, per questo teorema $convex(X)$ è un poliedro, ovvero $convex(X) = \{Cx\leq d, x \geq 0 \}$ che è anche la formulazione ideale.

La formulazione ideale gode di una \textbf{proprietà fondamentale}: se massimizzo o minimizzo $c^Tx$ utilizzando la formulazione ideale ($convex(X)$) ottengo sempre una soluzione ottima intera anche se risolvo il rilassamento continuo con il simplesso. Ovvero posso utilizzare variabili continue al posto di quelle intere, semplificando così il problema, e ottenere comunque una soluzione intera.

Serve quindi un modo per ottenere $convex(X)$, che non è cosa semplice (NP-Complete).













