\chapter{La nascita del copyright}

\section{Le origini}

Le prime forme di protezione in realtà non erano pensate per proteggere i diritti del beneficiario ma per dare dei vantaggi alle autorità che le emanavano; in secondo luogo non erano collegate alla conoscenza che raccoglieva quello si voleva proteggere, non si proteggeva il contenuto del libro ma si proteggeva l'ente industriale che lo aveva prodotto. Infine non erano collegate nemmeno all'autore. La forma era molto diversa da quella odierna.

Il copyright si è sviluppato in origine a Venezia intorno al 1469, 13 anni dopo la produzione della bibbia di Gutemberg. Prima dell'invenzione della stampa non c'era un sistema ben strutturato e organizzato, il libri costavano moltissimo, richiedevano anni di lavoro e di conseguenza non ve ne erano molti. 
La scrittura di un libro era un processo molto impegnativo, il costo di produzione stimato è intorno agli 80.000 \euro{} di oggi. 
Nel 1450 la bibbia di Gutemberg cambia decisamente le carte in gioco, viene creato un processo industriale della scrittura, che prima era quasi un'``opera d'arte''. 
Lo stato incominciò ad interessarsene, per \textbf{controllare il flusso di informazioni} ed \textbf{imporre dei blocchi sulla conoscenza}; questa fu la direzione presa in Inghilterra. 
Dall'altro lato la stampa era un'invenzione fenomenale e si voleva trarre vantaggi da essa; questa fu la direzione presa a Venezia, i quali erano molto interessati ad avere il sistema di stampa e a sfruttarlo; cercarono di fare in modo che tanta gente ce l'avesse, imponendo comunque dei controlli. 
In quest'ottica il copyright non nasce come un diritto, ma come forma di privilegio che l'autorità concede, ha una forma di incentivo brevettuale.

In Italia esistevano delle \textbf{corporazioni}, che detenevano il controllo sulla conoscenza delle arti artigiane, vi era tutto un sistema di privilegi ed erano loro a mantenere l'ordine.
Dall'altra parte gli stessi comuni che avevano creato queste corporazioni avevano anche creato un sistema per incentivare la gente degli altri comuni di svelare la conoscenza e diffondere le tecniche più avanzate. 

Quando nel 1469 Johannes of Speyer andò a Venezia chiese una forma di incentivo per portare la propria macchina di stampa a Venezia, e ovviamente glielo concessero, perché era una macchina importante dalle grandi potenzialità. Gli diedero dunque un'esclusiva sulla stampa per 5 anni. 
Al giorno d'oggi quello che conta di un libro è il suo contenuto, non la forma; all'epoca si pagavano i libri in funzione del peso o come merce di scambio. Pochi mesi dopo questa esclusiva però Johannes morì, e questo privilegio durò dunque per pochi mesi, e presto si cominciò dunque a formare un mercato sulla stampa. 

Una cosa importante che fu conseguenza di questo privilegio fu che il controllo della stampa venne sottratto alle corporazioni, la produzione si orientò dunque in un certo modo, non ci fu la differenziazione del modo in cui venivano gestiti i privilegi di stampa. Questi privilegi erano concessi volta per volta alle singole persone ed erano associati al modo in cui venivano prodotti i libri:

\begin{itemize}

\item Privilegi e non diritti d'autore; 
\item Era lo stampatore e non l'autore ad avere i diritti, anche perché spesso le opere non avevano un singolo autore non ben definito
\item Carattere tecnologico dei privilegi iniziali, non associati al contenuto. Ad esempio il privilegio per la stampa in italico di Manutius o quello per la miglioria della stampa di musica di Petrucci.

\end{itemize}

Questa gestione dei privilegi viene mantenuta anche dalla corporazione degli stampatori che nasce nel 1549.

Lo \textbf{statuto del 1474 dei privilegi} rappresenta un tentativo di portare ordine al sistema di gestione dei privilegi, andando a codificare le varie pratiche gestionali ed è mirato principalmente agli inventori piuttosto che alle corporazioni.

%Le opere all'epoca erano per la maggior parte diverse edizioni delle opere classiche.

%Uno \textbf{statuto} importantissimo, quello del 1474, per la prima volta stabiliva che quando una persona produceva qualche cosa di meritevole, di nuovo e originale, aveva diritto ad una protezione per 10 anni. Questa sembra per la prima volta una forma di protezione collegata alla proprietà intellettuale di quello che ci sta dentro e non esclusivamente ad un processo industriale. Era una cosa che si avvicinava a un \textit{diritto}. Ma di fatto questo statuto finì in un binario morto ma ebbe un effetto molto importante, perché si spostò l'attenzione per la prima volta dall'interesse degli stampatori agli \textbf{autori}. 

\section{Il rinascimento}

Il 1517 segna un cambiamento nel modo con cui vengono distribuiti i privilegi. Prima i privilegi venivano distribuiti in modo indipendente dal contenuto, quello che interessava era esclusivamente il processo di stampa. A un certo punto quando ci si stanca di avere tutti libri uguali della stessa opera, volevano avere libri un po' più \textit{nuovi}, e quindi vennero ritirati tutti i privilegi sui libri comuni in stampa, facendo così cadere le opere nel pubblico dominio e venne limitata la durate dei privilegi a 10 anni.

Fu necessario dunque lo spostamento del mercato verso le opere originali, le quali erano proteggibili. Per una volta quello che conta non è il modo in cui viene stampato un libro ma quello che ci sta dentro. Gli \textbf{autori} cominciarono ad avere dunque un po' più di potere. Questa protezione sulle opere si rafforzò nel tempo, andando a proteggere anche le \textbf{modifiche} sulle opere. Come conseguenza di questo i privilegi cominciarono ad essere garantiti anche agli autori.  

La regolamentazione delle arti artigiane era fatta dalle corporazioni, ma piano piano sempre più persone le stavano trovando più adatte. Si stava entrando nel rinascimento, un'epoca in cui si da più spazio all'uomo e alla sua creatività. Le corporazioni avevano il compito di regolamentare le arti, di proteggerle; ogni comune aveva le proprie. Una conseguenza importante di ciò fu la nascita del concetto di ``\textbf{proprietà immateriale}''. Si aveva la netta sensazione che quello che una persona conosceva era importante. D'altra parte la forma di protezione era strettamente legata alla comunità e la comunità va protetta proteggendo l'informazione. Questa forma di privilegio era molto legata agli autori e non più ai produttori. Si spostò l'interesse dal processo di produzione del libro al suo contenuto e in particolare all'autore.

Il sistema brevettuale parallelamente collegò il concetto di proprietà immateriale alla persona, anche perché questo è un periodo in cui c'è uno spostamento generale dalla comunità alla persona (umanesimo). Questo ebbe un grosso impatto. Prima chi gestiva la conoscenza erano degli artigiani, con la nascita di un interesse culturale diventa più ``teorico'', nasce una differenza tra la proprietà intellettuale e i suoi prodotti. Questo venne rafforzato ulteriormente dalla nascita degli \textbf{scrittori di professione}. Il valore delle opere deriva dall'individuo e dalle sue conoscenze.

\section{La situazione in Inghilterra}

Nel 1476 viene fondata la prima stamperia a Londra, anche se il fenomeno della stampa rimane inizialmente contenuto. Non veniva quindi dato peso ai problemi legati alla copia non autorizzata e solo in un secondo momento è stato introdotto un sistema di privilegi di stampa che venivano emessi dalla corona. 

Nel 1538 tutti i libri dovevo essere approvati dal consiglio prima della pubblicazione, tuttavia questo esponeva troppo la corona e quindi nel 1557 viene dato il controllo esclusivo sulla stampa dei libri alla Stationers Company (Corporazione dei librai di Londra), la quale si occupava dell'assegnazione dei monopoli di stampa, mentre la censura delle opere veniva svolta dalla \textbf{camera stellata}, una sorta di tribunale ``fittizzio'' che permetteva di controllare ciò che veniva espresso dalla gente.

Nel 1640 viene abolita la camera stellata e a quel punto ci fu la necessità di sostituire questa forma censoria. Questi controlli vennero affidati sempre alla Stationes Company nel 1643/1644, che esercitò un rigido controllo pre-stampa, stabilendo anche chi aveva il diritto di stampare. Naturalmente il diritto lo aveva solo chi si dimostrava premuroso nei confronti dei diritti della corona. Questa legge venne prorogata diverse volte fino al 1695. 

In quel periodo per poter stampare un'opera, questa doveva essere registrata nel \textbf{registro della corporazione} che era mantenuto dalla Stationers Company. La registrazione era effettuabile solamente se l'opera superava il controllo del censore della corona o della corporazione.

Una volta registrata l'opera nel registro, questa veniva inserita sotto il nome di uno dei membri della corporazione il quale ne acquisiva il \textbf{copyright}, ovvero il diritto esclusivo di pubblicarla.
Il copyright nasce quindi come diritto specifico dell'editore, diritto sul quale il reale autore non può quindi recriminare alcunché né guadagnare di conseguenza.

Verso la fine del XVII secolo, l'imporsi delle idee liberali nella società frenò le politiche censorie e causò una graduale fine del monopolio degli editori.

La soluzione fu di dare una licenza agli \textbf{autori}, dare loro una certa libertà, con lo \textbf{Statuto di Anna} del 1710.  Una volte registrato il libro presso la Stationers Company, l'autore deteneva il copyright sulla sua opera e poteva scegliere a chi dare la possibilità di stamparla.
Il monopolio dell'autore aveva una durata di 14 anni, con la possibilità di chiederne ulteriori 14.

Questo è il primo vero esempio di copyright, in quanto protegge le creazioni intangibili e viene dato agli \textbf{autori} e non alle stamperie, come accadeva prima. Il tutto con lo scopo di stimolare la cultura.

\section{Il nuovo mondo}

In America invece ci fu un approccio un po' misto tra quello sviluppatosi in Inghilterra e a Venezia. L'America era comunque una colonia inglese quindi risentiva in modo molto forte delle censure dell'Inghilterra. Nel 1638 il reverendo Glover porta la prima macchina a stampa in Massachussets, con lo scopo di divulgare il vangelo. Vi era un mix di controllo e di patrocinio.  Il primo privilegio di stampa è del 1672, si offriva di stampare qualcosa che fosse nell'interesse della comunità, e in cambio si chiedeva un aiuto. Questo fu un esempio di privilegio molto simile a quello di Venezia. 

Andrew Law fu il primo ad ottenere l'accordo sul privilegio d'autore nel 1781. Aveva paura che il suo stampatore gli fregasse il lavoro, quindi chiese ed ottenne questo privilegio legato al contenuto dell'opera. Si arriva ad una vera e propria forma di copyright nel 1783, in cui John Ledyard aveva chiesto di avere un privilegio di stampa; la cosa nuova è che la commissione che doveva decidere se concedere o meno questo privilegio raccomandò la definizione di un regolamento, venne quindi creato il \textbf{Connecticut copyright statute}. Nel 1790 questo decreto venne poi trasformato in un \textbf{Copyright act} che valeva in senso generale.

\section{Il copyright moderno}

Nel 1883 ci fu la \textbf{convenzione di Berna} che stabilì per la prima volta una regolamentazione internazionale, ragione per cui adesso è possibile parlare di copyright in senso generale. Infatti, prima della convenzione, le varie nazioni non si conoscevano reciprocamente i copyright. 

Inizialmente la convenzione prevedeva la tutela automatica dei diritti d'autore, senza necessità di una registrazione, la tutela dei diritti fino a 50 anni dopo la morte dell'autore, con l'obbligo di riconoscimento reciproco delle opere protette.

La convenzione ha poi subito una serie di modifiche nel tempo, fino al 1979, e attualmente include 165 paesi. 

Negli USA, prima dell'entrata in vigore del Copyright Act nel 1976, un'opera era protetta per 14 anni, trascorsi i quali l'opera cadeva nel pubblico dominio. Con l'approvazione del Copyright Act, la durata del copyright venne estesa a 50 anni dalla morte dell'autore (75 se appartenente ad un'impresa). Successivamente la durata del copyright venne ulteriormente estesa con il Copyright Term Extension Act (SonnyBono Act o Mickey Mouse Protection Act) che estendeva di ulteriori 20 anni il diritto d'autore sulle opere pubblicate dopo il 1923. La legge fu fortemente sostenuta da Sonny Bono, cantante e membro repubblicano del congresso, che morì prima dell'approvazione della legge e dalla Disney che l'appoggiò per evitare che Topolino cadesse nel dominio pubblico.

\section{Il copyright sul software}

Finora abbiamo parlato di copyright relativo alla carta stampata. Quando si parla di copyright sul \textbf{software} le cose cambiano drasticamente, perché non è molto chiaro se ciò che risiede sulla memoria di un computer possa essere proteggibile dal copyright. 
Oggi noi lo diamo per scontato ma libro e software hanno proprietà molto diverse. Il software non è una cosa tangibile, è qualcosa di ``nascosto''. L'idea fondamentale della legge sul copyright è che esso serve essenzialmente per proteggere la comunità, in modo da incentivare l'autore a condividere il proprio lavoro. 

Nel 1950 nascono i primi computer. Da lì non ci fu la necessità immediata di proteggere il software, perché di computer ce n'erano molto pochi, costavano molto e i lavori erano commissionati. Il codice sorgente una volta utilizzato veniva ``buttato via''. 
Si inizia a pensare di proteggere il software nel 1964, anno in cui un gruppo di \textbf{studenti} provò a vedere se fosse possibile farlo. Questo perché con il Copyright Act del 1909 si imponeva la registrazione del materiale protetto da copyright e non era ben chiaro come questo concetto si applicava al software.

Nel 1976, con il Copyright Act, e successivamente nel 1980 con il Computer Software Copyright Act risultò chiaro che il software era proteggibile. Quest'ultimo prevedeva:

\begin{itemize}
	\item La possibilità di effettuare una copia o l'adattamento del software solamente allo scopo di effettuare un back-up o se la modifica era necessaria per il corretto utilizzo del software.
	\item Che le copie esatte potessero essere vendute, a patto che venissero trasferite completamente al nuovo proprietario
	\item Che le copie modificate potessero essere utilizzate solo dal proprietario della copia.
\end{itemize}

Nel 1990 venne poi limitato il diritto di prima vendita, solo per uso non commerciale.

\section{Il diritto d'autore in Italia}

Diritto esclusivo dell'autore su:

\begin{itemize}

\item Ridistribuzione;
\item Modifica;
\item Adattamento;
\item Traduzione.

\end{itemize}

Tale diritto è \textbf{rinunciabile} e \textbf{trasferibile}.
